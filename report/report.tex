%%% Local Variables:
%%% mode: latex
%%% TeX-master: t
%%% End:
\documentclass[12pt, letterpaper]{article}

\usepackage[T1]{fontenc}
\usepackage[utf8]{inputenc}
\usepackage[frenchb]{babel}
\usepackage{gensymb}
\usepackage{latexsym}
\usepackage{titlesec}
\usepackage{marvosym}
\usepackage{enumitem}
\usepackage[pdftex, hidelinks]{hyperref}
\usepackage{graphicx}
\usepackage{rotating}
\usepackage{caption}
\usepackage{makecell}
\usepackage{adjustbox}
\usepackage{placeins}

\newcommand{\fncell}[3][]{\makecell[l]{#1 \\ #2 \\ #3}}
\newcommand{\PolOne}[2]{0.5$sin$({#1}) - 2$cos$({#1}) + $sin$({#2}) - 1.5$cos$({#2})}
\newcommand{\PolTwo}[2]{1.5$sin$({#1}) - $cos$({#1}) + 2$sin$({#2}) - 0.5$cos$({#2})}

\frenchbsetup{StandardItemLabels=true}

\author{Edorh François (EDOF19059507), Guison Vianney (GUIV30069402)}
\title{Rapport Projet de Métaheuristique en Optimisation (Sujet 1)}

\begin{document}
\maketitle
\tableofcontents
\newpage

\section{Introduction}
Nous avons choisi de comparer deux algorithmes génétiques, PESA et
IBEA, dans le cadre de l'optimisation multi-objectifs.\\

Le principe de chaque algorithme est rapidement évoqué, ainsi que des
exemples de solutions obtenues sur différents problèmes d'optimisation.\\

Pour finir, une comparaison des différents algorithmes est effectuée.

\newpage

\section{PESA}

\subsection{Principe}

À l'instar de SPEA2, PESA utilise une archive. Celle-ci contient les
meilleurs individus rencontrés durant l'exécution de l'algorithme (un
paramètre de configuration définit sa taille). Les solutions
correspondent au contenu de cette archive à la dernière itération.\\

Pour remplir l'archive, on récupère les individus non dominés de la
population actuelle. Les individus non dominés par le contenu de
l'archive sont ensuite ajouté un à un à l'intérieur. Les éléments de
l'archive dominés par ce nouvel individu en sont retirés. De plus, si
l'archive est temporairement en surcharge suite à cet ajout, un
élément de l'archive parmis ceux ayant le \textit{squeeze factor}
(défini plus bas) le plus élevé est retiré.\\

Pour calculer le \textit{squeeze factor}, on divise l'espace objectif
en une hyper-grille de taille prédéfinie (paramètre de configuration
spécifiant le nombre de cases par dimension). On assigne ensuite
chaque individu à une case de cette hyper-grille en fonction de ses
valeurs objectives. Pour chaque individu, son \textit{squeeze factor}
correspond au nombre d'invidus contenus dans la
même case que lui.\\
De plus, ce \textit{squeeze factor} est aussi utilisé lors de la sélection:
plus il est élevé, moins l'individu est intéressant.\\

Lors de la création d'une nouvelle population (une fois que l'archive
a été mise à jour), il se passe la chose suivante:

\begin{itemize}

\item Un crossover se produit avec une probabilité $Pc$. On
  sélectionne deux individus de l'archive, par rapport à leur
  \textit{squeeze factor}, et on génère \textbf{un} enfant.\\

\item Si il n'y a pas de crossover, une mutation se produit
  (probabilité égale à $1 - Pc$). Dans ce cas, on sélectionne un
  individu de l'archive et on génère un nouvel individu par mutation
  de celui-ci (avec probabilité $Pm$).

\end{itemize}

Cela se répète jusqu'à ce que la nouvelle population contienne $N$
éléments.

\subsection{Résultats}
Les figures \ref{fig:pesa_sch_pareto} à \ref{fig:pesa_zdt6_pareto}
montrent des exemples d'exécution de l'algorithme sur différents
problèmes.

\section{IBEA}

\subsection{Principe}

IBEA se base sur un indicateur de qualité \textit{I} pour le processus
de sélection qui permet de comparer la qualité de 2 set de Pareto
l'un à l'autre, cet indicateur permet par la suite de calculer la \textit{fitness}
d'un individus par rapport à tous les autres.\\
\\
Contrairement à des algorithmes plus classique on peut adapter IBEA en fonction
des préférences de l'utilisateur en choisissant un indicateur différent, de plus il
ne nécessite pas de mécanisme pour conserver la diversité des populations.
\\
Concernant son implémentions, dans un premier temps on vas calculer l'indicateur de qualité \textit{I} pour chaque paire d'individus, ces indicateurs
nous permettent d'attribuer une \textit{fitness} à toute notre population.\\
Ensuite on passe au processus de sélection, qui consiste à retirer l'individu
avec la plus basse \textit{fitness} jusqu'à ce que la population actuelle fasse la taille de
la population initiale, il est à noter qu'à chaque retrait il faut recalculer la \textit{fitness}
de tous individus.\\
A ce niveau si les conditions d'arrêt ne sont pas atteintes on applique un tournois sur notre population pour créer une nouvelle population à laquelle on applique une mutation, enfin on ajoute notre nouvelle population à l'ancienne et on recommence l'algorithme.

\subsection{Résultats}
Les figures \ref{fig:ibea_sch_pareto} à \ref{fig:ibea_zdt6_pareto}
montrent des exemples d'exécution de l'algorithme sur différents
problèmes.

\section{IBEA Adaptatif}

\subsection{Principe}

L'un des principaux problèmes d'IBEA est que le calcul de \textit{fitness} demande un paramètre de mise à l'échelle $\kappa$ à déterminer en fonction de \textit{I} et du problème traité, or les résultats de \textit{I} peuvent être très étalés ce qui rend
le choix de $\kappa$ très fastidieux.\\

L'IBEA adaptatif résout ce problème en mettant à l'échelle tout les points de la population dans un intervalle entre [-1,1] ce qui
permet d'utiliser le même $\kappa$ pour tout les problèmes.

\subsection{Résultats}
Les figures \ref{fig:pesa_sch_pareto} à \ref{fig:pesa_zdt6_pareto}
montrent des exemples d'exécution de l'algorithme sur différents
problèmes.


\section{Comparaison}

Les algorithmes ont été comparés en codage <XXX>. Les résultats à
chaque génération et ce sur 30 itérations ont été enregistrés et
comparés.\\
<Pour une itération et un problème donné, les algorithmes ont les mêmes
populations initiales.> => Vérifier

\subsection{Problèmes de test}
Les problèmes d'optimisation utilisés pour comparer les deux
algorithmes sont ceux utilisées pour évaluer l'algorithme NSGA2.\\
Les solutions optimales des problèmes KUR et POL n'ayant pu être
déterminées, les métriques associées n'ont pu être calculées. Ainsi,
pour ces deux problèmes, une comparaison visuelle du front de Pareto a
été faite.\\

\begin{table}[ht]
  \centering
  \begin{adjustbox}{width=1\textwidth}
    \begin{tabular}{|c|c|c|c|c|c|}
      \hline
      Problème & $n$ & Domaine de définition & Fonctions objectifs & Solutions optimales & Commentaires\\
      \hline
      SCH & 1 & $[-10^3, 10^3]$ & \fncell[$f_1(x) = x^2$]{$f_2(x) = (x - 2)^2$} & $x \in{} [0, 2]$ & convexe\\
      \hline
      FON & 3 & $[-4, 4]$ & \fncell{$f_1(x) = 1 - \exp{(-\sum_{i = 1}^{3}{(x_i - \frac{1}{\sqrt{3}})^2})}$} & \makecell{$x_1 = x_2  = x_3$ \\ $\in{} [\frac{-1}{\sqrt{3}}, \frac{1}{\sqrt{3}}]$} & non convexe\\
      \hline
      POL & 2 & $[-\pi, \pi]$ & \makecell{
                                \fncell{
                                \fncell{$f_1(x) = 1  + (A_1 - B_1)^2 + (A_2 - B_2)^2$}{$f_2(x) = (x_1 + 3)^2 + (x_2 + 1)^2$}}{
                                \fncell{$A_1(x) = \PolOne{1}{2}$}{$A_2(x) = \PolTwo{1}{2}$}} \\
      \fncell{$B_1(x) = \PolOne{x_1}{x_2}$}{$B_2(x) = \PolTwo{x_1}{x_2}$}} &  & \makecell{non convexe, \\ discontinu}\\
      \hline
      KUR & 3 & $[-5, 5]$ & \fncell[$f_1(x) = \sum_{i = 1}^{n - 1} (-10\exp{(-0.2\sqrt{x_i^2 + x_{i + 1}^2})})$]
                                   {$f_2(x) = \sum_{i = 1}^{n - 1} (|x_1|^{0.8} + 5 $sin$(x_i^3))$}&  & non convexe\\
      \hline
      ZDT1 & 30 & $[0, 1]$ & \fncell[$f_1(x) = x_1$]
                             {$f_2(x) = g(x) [1 - \sqrt{x_1 / g(x)}]$}
                             {$g(x) = 1 + 9(\sum_{i = 2}^{n} x_i) / (n - 1)$}
                                                                   & \makecell{$x_1 \in{} [0, 1]$ \\ $x_i = 0$, \\ $i = 2, ..., n$} & convexe\\
      \hline
      ZDT2 & 30 & $[0, 1]$ & \fncell[$f_1(x) = x_1$]
                             {$f_2(x) = g(x) [1 - (x_1 / g(x))^2]$}
                             {$g(x) = 1 + 9(\sum_{i = 2}^{n} x_i) / (n - 1)$}
                                                                   & \makecell{$x_1 \in{} [0, 1]$ \\ $x_i = 0$, \\ $i = 2, ..., n$} & non convexe\\
      \hline
      ZDT3 & 30 & $[0, 1]$ & \fncell[$f_1(x) = x_1$]
                             {$f_2(x) = g(x) [1 - \sqrt{x_1 / g(x)} - \frac{x_1}{g(x)}$sin$(10\pi{}x_1)]$}
                             {$g(x) = 1 + 9(\sum_{i = 2}^{n} x_i) / (n - 1)$}
                                                                   & \makecell{$x_1 \in{} [0, 1]$ \\ $x_i = 0$, \\ $i = 2, ..., n$} & \makecell{convexe, \\ discontinu}\\
      \hline
      ZDT4 & 10 & \makecell{$x_1 \in{} [0, 1]$ \\ $x_i \in{} [-5, 5]$, \\ $i = 2, ..., n$}
               & \fncell[$f_1(x) = x_1$]
                 {$f_2(x) = g(x) [1 - \sqrt{x_1 / g(x)}]$}
                 {$g(x) = 1 + 10(n - 1) + (\sum_{i = 2}^{n} [x_i^2 - 10$cos$(4\pi{}x_i)])$}
                     & \makecell{$x_1 \in{} [0, 1]$ \\ $x_i = 0$, \\ $i = 2, ..., n$} & non convexe\\
      \hline
      ZDT6 & 10 & $[0, 1]$ & \fncell[$f_1(x) = 1 - \exp(-4x_1)$sin$^6(6\pi{}x_1)$]
                             {$f_2(x) = g(x) [1 - (f_1(x) / g(x))^2]$}
                             {$g(x) = 1 + 9[(\sum_{i = 2}^{n} x_i) / (n - 1)]^{0.25}$}
                                                                   & \makecell{$x_1 \in{} [0, 1]$ \\ $x_i = 0$, \\ $i = 2, ..., n$} & \makecell{non convexe, \\ non uniformément \\ réparti}\\
      \hline
    \end{tabular}
  \end{adjustbox}
  \caption{Problèmes utilisés}
\end{table}

\FloatBarrier

\subsection{Configurations}


% \begin{table}[ht]
%   \centering
%   \begin{tabular}{|l|l|}
%     \hline
%     \textbf{Propriété} & \textbf{Valeur}\\
%     \hline
%     $N$ & 100 \\
%     \hline
%     $G_{max}$ & 250 \\
%     \hline
%     $P_c$ & 0.9 \\
%     \hline
%     $P_m$ & $\frac{1}{n}$  \\
%     \hline
%     Crossover & SBX$-20$ \\
%     \hline
%     Mutation & Polynomial$-20$ \\
%     \hline
%   \end{tabular}
%   \caption{Configuration commune}
% \end{table}

\FloatBarrier

\subsection{Métriques}
Les deux indicateurs de mesure $\Upsilon$, pour la convergence et
$\Delta$, pour la diversité, sont issus de l'article de NSGA2.

\subsubsection{Convergence}
La convergence $\Upsilon$ correspond à la distance moyenne du front
de Pareto que l'algorithme a calculé par rapport au front de Pareto
optimal.\\


Plus $\Upsilon$ tend vers zéro, plus les solutions convergent vers
les solutions optimales.

\subsubsection{Diversité}
L'indicateur a été légèrement modifié. L'article précise qu'il faut
tracer une courbe parallèle au front de Pareto optimal. Les solutions
extrêmes obtenues sont alors utilisées pour calculer $d_f$ et
$d_l$. Par soucis de simplicité, $d_f$ et $d_l$ sont obtenues en
utilisant les solutions extrêmes du front de Pareto optimal lui
même. Cela biaise cependant légèrement cet indice en faveur d'un
algorithme qui converge mieux.\\

Plus $\Delta$ tend vers zéro, plus la diversité des solutions est
importante.

\begin{center}
  $\Delta = \frac{d_f + d_l + \sum_{i = 1}^{N - 1} |d_i - \bar{d}|}{d_f + d_l + (N - 1) \bar{d}}$
\end{center}

Avec $\bar{d}$, la distance consécutive moyenne des points.

\subsection{Résultats}

\newpage

\section{Conclusion}


\newpage

\section{Figures}

\FloatBarrier

\end{document}

%%% Local Variables:
%%% mode: latex
%%% TeX-master: t
%%% End:
